\documentclass[a4paper]{article}
\usepackage{a4wide}
\usepackage[utf8]{inputenc}
\usepackage[T1]{fontenc}
\usepackage{lmodern}
\usepackage{amsmath}
\usepackage{bbm}
\usepackage{hyperref}
\usepackage{framed}
\usepackage{stmaryrd}
\usepackage{xcolor}
\usepackage{placeins}

\newcommand{\version}{1.17}

\newcommand{\F}{\mathbbm{F}}
\newcommand{\G}{\mathbbm{G}}
\newcommand{\Z}{\mathbbm{Z}}
\newcommand{\N}{\mathbbm{N}}
\newcommand{\I}{\mathbbm{I}}
\newcommand{\B}{\mathbbm{B}}

\newcommand{\public}{\textsf{public}}
\newcommand{\shuffle}{\textsf{shuffle}}
\newcommand{\basesixfour}{\textsf{BASE64}}
\newcommand{\shatwo}{\textsf{SHA256}}

\newcommand{\jstring}{\texttt{string}}
\newcommand{\uuid}{\texttt{uuid}}
\newcommand{\tpk}{\texttt{trustee\_public\_key}}
\newcommand{\election}{\texttt{election}}
\newcommand{\ballot}{\texttt{ballot}}
\newcommand{\etally}{\texttt{encrypted\_tally}}
\newcommand{\pdecryption}{\texttt{partial\_decryption}}
\newcommand{\result}{\texttt{result}}

\newcommand{\cert}{\texttt{cert}}
\newcommand{\poly}{\texttt{polynomial}}
\newcommand{\vinput}{\texttt{vinput}}
\newcommand{\voutput}{\texttt{voutput}}

\newcommand{\vc}[1]{\textcolor{blue}{#1}}
\newcommand{\vcomment}[1]{\textcolor{violet}{#1}}

\title{Belenios specification}
\date{Version~\version}
\author{Stéphane Glondu}

\begin{document}
\maketitle
\tableofcontents

\section{Introduction}
{\it References.}
This document is a specification of the voting protocol implemented in
Belenios \version.
A high level description of Belenios and some statistics about its
usage can be found~\cite{Belenios-Meadows2019}.
A security proof of the protocol for ballot privacy and verifiability
is presented in~\cite{Belenios-Easycrypt-CSF18}. The proof has been conducted with the
tool EasyCrypt. It focuses on the protocol aspects and assumes
security of the cryptographic primitives.
The cryptographic primitives have been introduced in various places
and their security proofs is spread across several references.
\begin{itemize}
\item 
The threshold decryption scheme is
based on a ``folklore'' scheme:
Pedersen’s~\cite{Pedersen} Distributed Key Generation (DKG) that has several variations.
The variant considered in Belenios is described in~\cite{wpes2013} and
proved in~\cite{wpes2013,asiacrypt12}.
\item Ballots are composed of an ElGamal encryption of the votes and a
  zero-knowledge proof of well-formedness, as for the Helios
  protocol~\cite{Helios}. Compared to Helios, we support blank votes,
  which required to adapt the zero-knowledge proofs, as specified and
  proved in~\cite{note-Pierrick}. Additionally, ballots are signed to
  avoid ballot stuffing, as introduced in~\cite{CGGI-esorics14} and also
  described in~\cite{Belenios-Meadows2019}.
  \item During the tally phase, Belenios supports two modes. Ballots are either combined
    homomorphically or shuffled and randomized, using mixnets. The
    mixnet algorithms are taken from the CHVote specification~\cite{CHVote}.
\end{itemize}

% has been conducted with EasyCrypt and shows
% More discussion, theoretical explanations and
% bibliographical references can be found in an article
% available online.\footnote{\url{https://hal.inria.fr/hal-02066930/document}}

{\it Types of supported elections.}
Belenios supports two main types of questions.
In the \emph{homomorphic case}, voters can select between $k_1$ and
$k_2$ candidates out of $k$ candidates. This case is called
homomorphic because the result of the election for such questions is
the number of votes received for each candidate. No more information
is leaked.
In the \emph{non-homomorphic case}, voters can give a number to each
candidate. This can be used to rank candidates or grade them. Then the
(raw) result of the election is simply the list of votes, as emitted
by the voters, in
a random order, to preserve privacy.
Any counting method can then be  applied
(e.g. Condorcet, STV, or majority judgement) although Belenios does
not offer support for this.
The non-homomorphic case  therefore offers much more flexibility, at
the cost of extra steps during the tally (in order to securely shuffle
the ballots).
Belenios supports both types of questions and an election can even
mix homomorphic and non-homomorphic questions.
% and slightly less privacy
\medskip

{\it Group parameters.}
The cryptography involved in Belenios needs a cyclic group $\G$ where
discrete logarithms are hard to compute. We will denote by $g$ a
generator and $q$ its order. We use a multiplicative notation for the
group operation. For practical purposes, we use a multiplicative
subgroup of $\F^*_p$ (hence, all exponentiations are implicitly done
modulo $p$). We suppose the group parameters are agreed on
beforehand. Default group parameters are given as examples in
section~\ref{default-group}.

\medskip

{\it Weights.}  In the homomorphic case (and only in the homomorphic
case), each voter has a weight: a
ballot is counted as many times as the weight of its owner. Usually,
the weight of all voters is 1 but sometimes, it may be useful to
assign different weights. We assume the sum of all weights is not too
big, so that it can be computed as the discrete logarithm of some
group element.

\section{Parties}

\newcommand{\pk}{\texttt{public\_key}}
\newcommand{\sk}{\texttt{private\_key}}
\newcommand{\proof}{\texttt{proof}}
\newcommand{\iproof}{\texttt{iproof}}
\newcommand{\ciphertext}{\texttt{ciphertext}}

\newcommand{\pklabel}{\textsf{public\_key}}
\newcommand{\pok}{\textsf{pok}}
\newcommand{\challenge}{\textsf{challenge}}
\newcommand{\response}{\textsf{response}}
\newcommand{\alphalabel}{\textsf{alpha}}
\newcommand{\betalabel}{\textsf{beta}}
\newcommand{\Hash}{\mathcal{H}}

\begin{itemize}
\item $\mathcal{A}$: server administrator
\item $\mathcal{C}$: credential authority
\item $\mathcal{T}_1,\dots,\mathcal{T}_m$: trustees
\item $\mathcal{V}_1,\dots,\mathcal{V}_n$: voters; each voter has a
    weight $w_i$ equal to 1 by default
%\item $\mathcal{M}_1,\dots,\mathcal{M}_p$: shufflers (if using non-homomorphic questions)
\item $\mathcal{S}$: voting server \\
  The voting server maintains the public data $D$ that
consists of:
  \begin{itemize}
  \item the election data $E$
  \item the structure $PK$ that contains the verification keys of the
    trustees and other verification material
  \item the list $L$ of public credentials
  \item the list $B$ of accepted ballots
  \item the result of the election {\result} (once the election is tallied)
  \end{itemize}
\end{itemize}

\section{Processes}
\label{processes}

\subsection{Election setup}
\label{election-setup}

\begin{enumerate}
\item $\mathcal{A}$ generates a fresh \hyperref[basic-types]{$\uuid$} $u$ and
  sends it to $\mathcal{C}$
\item $\mathcal{C}$ generates \hyperref[credentials]{credentials}
  $c_1,\dots,c_n$ and computes
  \[L=\shuffle((\public(c_1),w_1),\dots,(\public(c_n),w_n))\]
\item for $j\in[1\dots n]$, $\mathcal{C}$ sends $c_j$ to $\mathcal{V}_j$
\item \label{item-forget} (optional) $\mathcal{C}$ forgets $c_1,\dots,c_n$
 \item $\mathcal{C}$ sends $L$ to $\mathcal{A}$
 \item $\mathcal{A}$ checks that the multiset of weights in $L$ is the
   same as $\{w_1,\dots,w_n\}$
 \item $\mathcal{A}$ defines the shape of the
   \hyperref[trustees]{$\texttt{trustees}$} structure that will be
   used in the election;
 \item $\mathcal{A}$ and $\mathcal{T}_1,\dotsc,\mathcal{T}_m$ run key
   establishment protocols (see \ref{process-filling-trustees}) as
     needed to fill in the $\texttt{trustees}$ structure;
\item $\mathcal{A}$ creates the \hyperref[elections]{$\election$} $E$
\item $\mathcal{A}$ loads $E$ and $L$ into $\mathcal{S}$ and starts it
\item $\mathcal{C}$ checks that the list of public credentials $L$
  is exactly the one that appears on the election data of the election of
  {$\uuid$} $u$.
\end{enumerate}
Step~\ref{item-forget} is optional. It offers a better protection
against ballot stuffing in case $\mathcal{C}$ unintentionally leaks
private credentials.

\subsubsection{Filling in the \texttt{trustees} structure}
\label{process-filling-trustees}

The \hyperref[trustees]{\texttt{trustees}} structure consists of
\texttt{"Single"} or \texttt{"Pedersen"} items. For each of these
items, one or several trustees run the corresponding protocol below to
produce a sub-key $y_\tau$. Once all protocols have been run,
$\mathcal{A}$ synthesizes the global election public key $y$ from the
sub-keys computed in each protocol by multiplying them:
\[
  y=\prod_\tau y_\tau
\]

\paragraph{\texttt{"Single"} protocol}

This protocol involves a single trustee $\mathcal{T}$, whose presence
will be required to compute the tally.
\begin{enumerate}
\item $\mathcal{T}$ generates a \hyperref[trustee-keys]{$\tpk$}
  $\gamma$ and sends it to $\mathcal{A}$
\item $\mathcal{A}$ checks $\gamma$
\end{enumerate}
Later, when the election is open:
\begin{enumerate}
\item $\mathcal{T}$ checks that $\gamma$ appears in the set of verification
  keys $PK$ of the election of {$\uuid$} $u$ (the id of the election
  should be publicly known)
\end{enumerate}
The sub-key for this protocol is the $\textsf{public\_key}$ field of
$\gamma$.

\paragraph{\texttt{"Pedersen"} protocol}

This protocol involves $\mu$ trustees
$\mathcal{T}_1,\dots,\mathcal{T}_\mu$ such that only a subset of
$t+1$ of them will be needed to compute the tally.
\begin{enumerate}
\item for $z\in[1\dots \mu]$,
  \begin{enumerate}
  \item $\mathcal{T}_z$ generates a \hyperref[certificates]{$\cert$} $\gamma_z$
    and sends it to $\mathcal{A}$
  \item $\mathcal{A}$ checks $\gamma_z$
  \end{enumerate}
\item $\mathcal{A}$ assembles $\Gamma=\gamma_1,\dotsc,\gamma_\mu$
\item for $z\in[1\dots \mu]$,
  \begin{enumerate}
  \item $\mathcal{A}$ sends $\Gamma$ to $\mathcal{T}_z$ and $\mathcal{T}_z$ checks it
  \item $\mathcal{T}_z$ generates a \hyperref[polynomials]{$\poly$} $P_z$ and
    sends it to $\mathcal{A}$
  \item $\mathcal{A}$ checks $P_z$
  \end{enumerate}
\item for $z\in[1\dots \mu]$, $\mathcal{A}$ computes a
  \hyperref[vinputs]{$\vinput$} $\textsf{vi}_z$
\item for $z\in[1\dots \mu]$,
  \begin{enumerate}
  \item $\mathcal{A}$ sends $\Gamma$ to $\mathcal{T}_z$ and $\mathcal{T}_z$ checks it
  \item $\mathcal{A}$ sends $\textsf{vi}_z$ to $\mathcal{T}_z$ and $\mathcal{T}_z$ checks it
  \item $\mathcal{T}_z$ computes a \hyperref[voutputs]{$\voutput$} $\textsf{vo}_z$ and
    sends it to $\mathcal{A}$
  \item $\mathcal{A}$ checks $\textsf{vo}_z$
  \end{enumerate}
\item $\mathcal{A}$ extracts encrypted decryption keys $K_1,\dots,K_\mu$ and
  \hyperref[threshold-params]{threshold parameters}
\end{enumerate}
Later, when the election is open:
\begin{enumerate}
\item for $z\in[1\dots \mu]$, $\mathcal{T}_z$ checks that $\gamma_z$
  appears in the set of verification keys $PK$ of the election of {$\uuid$}
  $u$ (the id of the election should be publicly known).
\end{enumerate}
The sub-key for this protocol is computed from the polynomials of each
trustee as specified in section~\ref{polynomials}.

\subsection{Vote}

\begin{enumerate}
\item $\mathcal{V}$ gets $E$
\item $\mathcal{V}$ creates a \hyperref[ballots]{$\ballot$} $b$ and submits it to $\mathcal{S}$
\item $\mathcal{S}$ validates $b$ and adds it to $B$
\item at any time (even after tally), $\mathcal{V}$ may check that $b$
  appears in the list of accepted ballots $B$ and the weight of her
  ballot as it appears in $B$ is equal to her weight
\end{enumerate}


\subsection{Credential recovery}

If $\mathcal C$ has forgotten the private credentials of the voter
(optional step~\ref{item-forget} of the setup) then credentials cannot
be recovered.

If $\mathcal C$ has the list of private credentials (associated to the
voters), credentials can be recovered:
\begin{enumerate}
\item $\mathcal{V}_i$ contacts $\mathcal{C}$
\item $\mathcal{C}$ looks up $\mathcal{V}_i$'s private credential $c_i$
\item $\mathcal{C}$ sends $c_i$
\end{enumerate}

\subsection{Tally}

\begin{enumerate}
\item $\mathcal{A}$ stops $\mathcal{S}$ and computes the initial
  \hyperref[tally]{$\etally$} $\Pi_0$
\item $\mathcal{A}$ extracts the non-homomorphic ciphertexts from the
  encrypted tally (see
  section~\ref{shuffles}):
  \[\tilde\Pi_0=\textsf{nh\_ciphertexts}(\Pi_0)\]
\item if the election contains a non-homomorphic part, that is, if
  $\tilde\Pi_0\neq []$,
 then for $z\in[1\dots m]$:
  \begin{enumerate}
  \item $\mathcal{A}$ sends $\tilde\Pi_{z-1}$ to $\mathcal{T}_z$
  \item $\mathcal{T}_z$ runs the shuffle algorithm, producing a
    \hyperref[shuffles]{$\texttt{shuffle}$} $\sigma_z$ and sends it to
    $\mathcal{A}$
  \item $\mathcal{A}$ verifies $\sigma_z$ and extracts
    $\tilde\Pi_z=\textsf{ciphertexts}(\sigma_z)$
  \end{enumerate}
\item $\mathcal{A}$ merges shuffled non-homomorphic ciphertexts with
  homomorphic ciphertexts, i.e. builds $\Pi$ such
  that:\[\tilde\Pi_m=\textsf{nh\_ciphertexts}(\Pi)\]
\item for $z\in[1\dots m]$ (or, if in threshold mode, a subset of it
  of size at least $t+1$),
  \begin{enumerate}
  \item $\mathcal{A}$ sends $\Pi$ (and $K_z$ if in threshold mode) to
    $\mathcal{T}_z$
  \item $\mathcal{T}_z$ generates a \hyperref[tally]{$\pdecryption$} $\delta_z$
    and sends it to $\mathcal{A}$
  \item $\mathcal{A}$ verifies $\delta_z$
  \end{enumerate}
\item $\mathcal{A}$ combines all the partial decryptions, computes and publishes
  the election \hyperref[election-result]{\result}
\item $\mathcal{T}_z$ checks that $\delta_z$ and $\sigma_z$ appears in
  {\result}
\end{enumerate}

\subsection{Audit}

Belenios can be publicly audited: anyone having access to the (public)
election data can check that the ballots are well formed and that the
result corresponds to the ballots. Ideally, the list of ballots should
also be monitored during the voting phase, to guarantee that no ballot
disappears.

\subsubsection{During the voting phase}
\label{sec:audit-voting}
At any time, an auditor can retrieve the public board and check its consistency. She should
always record at least the last audited board. Then:
\begin{enumerate}
\item she retrieves the election data $D = (E,PK,L,B,r)$ where $B$ is the list of ballots;
  \begin{itemize}
  \item she records $D$;
  \item for $b\in B$, she checks that the proofs of $b$ are valid and that
  the signature of $b$ is valid and corresponds to one of the keys in
  $L$; she also checks that the weights correspond;
  \item she checks that any two ballots in $B$ correspond to distinct keys (of
    $L$);
  \end{itemize}
\item she retrieves the previously recorded election data $D' = (E',PK',L',B',r')$ (if it
  exists);
  \begin{itemize}
  \item for $b\in B'$, she checks that
    \begin{itemize}
    \item $b\in B$
    \item or $\exists b'\in B$ such that $b$ and $b'$ correspond to
      the same key in $L$. This corresponds to the case where a voter
      has revoted;
    \end{itemize}
    \item she checks that all the other data is unchanged: $E=E'$, $PK=PK'$, $L=L'$,
      and $r=r'$ (actually the result is empty at this step).
\end{itemize}
\end{enumerate}

There is no tool support on the web interface for these checks,
instead the command line tool \texttt{verify-diff} can be used.

\subsubsection{After the tally}
The auditor retrieves the election data $D$ and in
  particular the list $B$ of ballots and the
  \hyperref[election-result]{\result} $r$. Then:

  \begin{enumerate}
  \item she checks consistency of $B$, that is, performs all
    the checks described at step 1 of section~\ref{sec:audit-voting};
  \item she checks that $B$ corresponds to the board
      monitored so far, thus performs all
    the checks described at step 2 of section~\ref{sec:audit-voting};
  \item she checks that the proofs of the result $r$ are valid
    w.r.t. $B$.
She checks in particular the proofs of correct decryption and the
proofs of correct shuffling (when shufllers have been used).
\end{enumerate}
To ease verification of the trustees and the credential authorities,
it is possible to display the hash of their public data (e.g. the
public keys and the partial decryptions of the trustees, the hash of
the list of the public credentials) in some human-readable form. In
that case, the audit should also check that this human-readable data is
consistent with the election data.

There is no tool support on the web interface for these checks,
instead the command line tool \texttt{verify} can be used.

\section{Messages}
\label{messages}

\subsection{Conventions}

Structured data is encoded in JSON (RFC 4627). There is no specific
requirement on the formatting and order of fields, but care must be
taken when hashes are computed. We use the notation
$\textsf{field}(o)$ to access the field \textsf{field} of $o$.

\subsection{Basic types}
\label{basic-types}

\begin{itemize}
\item $\jstring$: JSON string
\item $\uuid$: election identifier (a string of
  Base58 characters\footnote{Base58 characters are:
    \texttt{123456789ABCDEFGHJKLMNPQRSTUVWXYZabcdefghijkmnopqrstuvwxyz}}
  of size at least 14), encoded as a JSON string
\item $\I$: small integer, encoded as a JSON number
\item $\B$: boolean, encoded as a JSON boolean
\item $\N$, $\Z_q$, $\G$: big integer, written in base 10 and encoded as a
  JSON string
\end{itemize}

\subsection{Common structures}
\label{common}

\begin{gather*}
  \proof=\left\{
    \begin{array}{rcl}
      \challenge&:&\Z_q\\
      \response&:&\Z_q
    \end{array}
  \right\}
  \qquad
  \ciphertext=\left\{
    \begin{array}{rcl}
      \alphalabel&:&\G\\
      \betalabel&:&\G
    \end{array}
  \right\}
\end{gather*}

\subsection{Verification keys}
\label{trustee-keys}

\begin{gather*}
  \pk=\G\qquad\sk=\Z_q\\
  \tpk=\left\{
    \begin{array}{rcl}
      \pok&:&\proof\\
      \pklabel&:&\pk
    \end{array}
  \right\}
\end{gather*}

A private key is a number $x$ modulo $q$, chosen at random in the
basic decryption mode, and computed after several interactions in the
threshold mode.
The corresponding
$\pklabel$ is $X=g^x$. A $\tpk$ is a bundle of this public key with a
\hyperref[common]{$\proof$} of knowledge computed as follows:
\begin{enumerate}
\item pick a random $w\in\Z_q$
\item compute $A=g^w$
\item $\challenge=\Hash_\pok(X,A)\mod q$
\item $\response=w+x\times\challenge\mod q$
\end{enumerate}
where $\Hash_\pok$ is computed as follows:
\[\Hash_\pok(X,A) = \shatwo(\verb=pok|=X\verb=|=A) \]
where $\pok$ and the vertical bars are verbatim and numbers are
written in base 10. The result is interpreted as a 256-bit big-endian
number. The proof is verified as follows:
\begin{enumerate}
\item compute $A={g^\response}/{X^\challenge}$
\item check that $\challenge=\Hash_\pok(X,A)\mod q$
\end{enumerate}

\subsection{Messages specific to threshold decryption support}
\label{threshold}

\subsubsection{Public key infrastructure}
\label{pki}

Establishing a public key so that threshold decryption is supported
requires private communications between trustees. To achieve this,
Belenios uses a custom public key infrastructure. During the key
establishment protocol, each trustee starts by generating a secret
seed (at random), then derives from it encryption and decryption keys,
as well as signing and verification keys. These four keys are then
used to exchange messages between trustees by using $\mathcal{A}$ as a proxy.

The secret seed $s$ is a 22-character string, where characters are
taken from the set:
\[\texttt{123456789ABCDEFGHJKLMNPQRSTUVWXYZabcdefghijkmnopqrstuvwxyz}\]

\paragraph{Deriving keys}

The (private) signing key $\textsf{sk}$ is derived by computing the
SHA256 of $s$ prefixed by the string \verb/sk|/. The corresponding
(public) verification key is $g^{\textsf{sk}}$. The (private)
decryption key $\textsf{dk}$ is derived by computing the SHA256 of $s$
prefixed by the string \verb/dk|/. The corresponding (public)
encryption key is $g^{\textsf{dk}}$.

\paragraph{Signing}

Signing takes a signing key $\textsf{sk}$ and a \textsf{message} $M$
(as a $\jstring$), computes a \textsf{signature} and produces a
$\texttt{signed\_msg}$. For the signature, we use a (Schnorr-like)
non-interactive zero-knowledge proof.

\begin{gather*}
  \texttt{signed\_msg}=\left\{
    \begin{array}{rcl}
      \textsf{message}&:&\jstring\\
      \textsf{signature}&:&\texttt{proof}
    \end{array}
  \right\}
\end{gather*}
To compute the \textsf{signature},
\begin{enumerate}
\item pick a random $w\in\Z_q$
\item compute the commitment $A=g^w$
\item compute the \textsf{challenge} as
  $\textsf{SHA256}(\texttt{sigmsg|}M\texttt{|}A)$, where $A$ is written
  in base 10 and the result is interpreted as a 256-bit big-endian
  number
\item compute the \textsf{response} as
  $w-\textsf{sk}\times\textsf{challenge}\mod q$
\end{enumerate}
To verify a \textsf{signature} using a verification key \textsf{vk},
\begin{enumerate}
\item compute the commitment $A=g^{\textsf{response}}\times\textsf{vk}^{\textsf{challenge}}$
\item check that $\textsf{challenge}=\textsf{SHA256}(\texttt{sigmsg|}M\texttt{|}A)$
\end{enumerate}

\paragraph{Encrypting}

Encrypting takes an encryption key $\textsf{ek}$ and a message $M$ (as
a $\jstring$), computes an \texttt{encrypted\_msg} and serializes it
as a $\jstring$. We use an El Gamal-like system.

\begin{gather*}
  \texttt{encrypted\_msg}=\left\{
    \begin{array}{rcl}
      \textsf{alpha}&:&\G\\
      \textsf{beta}&:&\G\\
      \textsf{data}&:&\jstring
    \end{array}
  \right\}
\end{gather*}
To compute the \texttt{encrypted\_msg}:
\begin{enumerate}
\item pick random $r,s\in\Z_q$
\item compute $\textsf{alpha}=g^r$
\item compute $\textsf{beta}=\textsf{ek}^r\times g^s$
\item compute $\textsf{data}$ as the hexadecimal encoding of the (symmetric)
  encryption of $M$ using AES in CCM mode with
  $\textsf{SHA256}(\texttt{key|}g^s)$ as the key and $\textsf{SHA256}(\texttt{iv|}g^r)$ as the
  initialization vector (where numbers are written in base 10)
\end{enumerate}
To decrypt an \texttt{encrypted\_msg} using a decryption key \textsf{dk}:
\begin{enumerate}
\item compute the symmetric key as $\textsf{SHA256}(\texttt{key|}\textsf{beta}/(\textsf{alpha}^{\textsf{dk}}))$
\item compute the initialization vector as $\textsf{SHA256}(\texttt{iv|}\textsf{alpha})$
\item decrypt $\textsf{data}$
\end{enumerate}

\subsubsection{Certificates}
\label{certificates}

A certificate is a \texttt{signed\_msg} encapsulating a serialized
\texttt{cert\_keys} structure, itself filled with the public keys
generated as described in section~\ref{pki}.
\begin{gather*}
  \texttt{cert}=\texttt{signed\_msg}
  \qquad
  \texttt{cert\_keys}=\left\{
    \begin{array}{rcl}
      \textsf{verification}&:&\G\\
      \textsf{encryption}&:&\G
    \end{array}
  \right\}
\end{gather*}
The message is signed with the signing key associated to
\textsf{verification}.

\subsubsection{Channels}
\label{channels}

A \textsf{message} is sent securely from \textsf{sk} (a signing key)
to \textsf{recipient} (an encryption key) by encapsulating it in a
\texttt{channel\_msg}, serializing it as a $\jstring$, signing it with
\textsf{sk} and serializing the resulting \texttt{signed\_msg} as a
$\jstring$, and finally encrypting it with \textsf{recipient}. The
resulting $\jstring$ will be denoted by
$\textsf{send}(\textsf{sk},\textsf{recipient},\textsf{message})$, and
can be transmitted using a third-party (such as the election
administrator).
\begin{gather*}
  \texttt{channel\_msg}=\left\{
    \begin{array}{rcl}
      \textsf{recipient}&:&\G\\
      \textsf{message}&:&\jstring
    \end{array}
  \right\}
\end{gather*}
When decoding such a message, \textsf{recipient} must be checked.

\subsubsection{Polynomials}
\label{polynomials}

Let $\Gamma=\gamma_1,\dotsc,\gamma_m$ be the certificates of all
trustees. We will denote by $\textsf{vk}_z$ (resp. $\textsf{ek}_z$)
the \textsf{verification} key (resp. the \textsf{encryption} key) of
$\gamma_z$. Each trustee must compute a \texttt{polynomial} structure
in step 3 of the key establishment protocol.
\begin{gather*}
  \texttt{polynomial}=\left\{
    \begin{array}{rcl}
      \textsf{polynomial}&:&\jstring\\
      \textsf{secrets}&:&\jstring^\ast\\
      \textsf{coefexps}&:&\texttt{coefexps}
    \end{array}
  \right\}
\end{gather*}
Suppose $\mathcal{T}_i$ is the trustee who is computing. Therefore, $\mathcal{T}_i$ knows
the signing key $\textsf{sk}_i$ corresponding to $\textsf{vk}_i$ and the
decryption key $\textsf{dk}_i$ corresponding to $\textsf{ek}_i$. $\mathcal{T}_i$
first checks that keys indeed match. Then $\mathcal{T}_i$ picks a random
polynomial
\[
  f_i(x)=a_{i0}+a_{i1}x+\dotsb+a_{it}x^t\in\Z_q[x]
\]
and computes $A_{ik}=g^{a_{ik}}$ for $k=0,\dotsc,t$ and
$s_{ij}=f_i(j)\mod q$ for $j=1,\dotsc,m$. $\mathcal{T}_i$ then fills the
\texttt{polynomial} structure as follows:
\begin{itemize}
\item the \textsf{polynomial} field is
  $\textsf{send}(\textsf{sk}_i,\textsf{ek}_i,M)$ where $M$ is a
  serialized \texttt{raw\_polynomial} structure
  \begin{gather*}
    \texttt{raw\_polynomial}=\left\{
      \begin{array}{rcl}
        \textsf{polynomial}&:&\Z_q^\ast
      \end{array}
    \right\}
  \end{gather*}
  filled with $a_{i0},\dotsc,a_{it}$
\item the \textsf{secrets} field is
  $\textsf{send}(\textsf{sk}_i,\textsf{ek}_1,M_{i1}),\dotsc,\textsf{send}(\textsf{sk}_i,\textsf{ek}_m,M_{im})$
  where $M_{ij}$ is a serialized \texttt{secret} structure
  \begin{gather*}
    \texttt{secret}=\left\{
      \begin{array}{rcl}
        \textsf{secret}&:&\Z_q
      \end{array}
    \right\}
  \end{gather*}
  filled with $s_{ij}$
\item the \textsf{coefexps} field is a signed message containing a
  serialized \texttt{raw\_coefexps} structure
  \begin{gather*}
    \texttt{coefexps}=\texttt{signed\_msg}
    \qquad
    \texttt{raw\_coefexps}=\left\{
      \begin{array}{rcl}
        \textsf{coefexps}&:&\G^\ast
      \end{array}
    \right\}
  \end{gather*}
  filled with $A_{i0},\dotsc,A_{it}$
\end{itemize}

The sub-key for this protocol run will be:
\[
y=\prod_{z\in[1\dots m]}g^{f_z(0)}=\prod_{z\in[1\dots m]}A_{z0}
\]

\subsubsection{Vinputs}
\label{vinputs}

Once we receive all the \texttt{polynomial} structures
$P_1,\dotsc,P_m$, we compute (during step 4) input data (called
\texttt{vinput}) for a verification step performed later by the
trustees. Step 4 can be seen as a routing step.
\begin{gather*}
  \texttt{vinput}=\left\{
    \begin{array}{rcl}
      \textsf{polynomial}&:&\jstring\\
      \textsf{secrets}&:&\jstring^\ast\\
      \textsf{coefexps}&:&\texttt{coefexps}^\ast
    \end{array}
  \right\}
\end{gather*}
Suppose we are computing the \texttt{vinput} structure $\textsf{vi}_j$
for trustee $\mathcal{T}_j$. We fill it as follows:
\begin{itemize}
\item the \textsf{polynomial} field is the same as the one of $P_j$
\item the \textsf{secret} field is
  $\textsf{secret}(P_1)_j,\dotsc,\textsf{secret}(P_m)_j$
\item the \textsf{coefexps} field is
  $\textsf{coefexps}(P_1),\dotsc,\textsf{coefexps}(P_m)$
\end{itemize}
Note that the \textsf{coefexps} field is the same for all trustees.

In step~5, $\mathcal{T}_j$ checks consistency of $\textsf{vi}_j$ by unpacking it
and checking that, for $i=1,\dotsc,m$,
\[
g^{s_{ij}}=\prod_{k=0}^t(A_{ik})^{j^k}
\]

\subsubsection{Voutputs}
\label{voutputs}

In step 5 of the key establishment protocol, a trustee $\mathcal{T}_j$ receives
$\Gamma$ and $\textsf{vi}_j$, and produces a \texttt{voutput}
$\textsf{vo}_j$.
\begin{gather*}
  \texttt{voutput}=\left\{
    \begin{array}{rcl}
      \textsf{private\_key}&:&\jstring\\
      \textsf{public\_key}&:&\texttt{trustee\_public\_key}
    \end{array}
  \right\}
\end{gather*}
Trustee $\mathcal{T}_j$ fills $\textsf{vo}_j$ as follows:
\begin{itemize}
\item \textsf{private\_key} is set to
  $\textsf{send}(\textsf{sk}_j,\textsf{ek}_j,S_j)$, where $S_j$ is $\mathcal{T}_j$'s
  (private) decryption key:
  \[
    S_j=\sum_{i=1}^m s_{ij}\mod q
  \]
\item \textsf{public\_key} is set to a
  \hyperref[trustee-keys]{\texttt{trustee\_public\_key}} structure
  built using $S_j$ as private key, which computes the corresponding
  public key and a proof of knowledge of $S_j$.
\end{itemize}
The administrator checks $\textsf{vo}_j$ as follows:
\begin{itemize}
\item check that:
  \[
    \textsf{public\_key}(\textsf{public\_key}(\textsf{vo}_j))=\prod_{i=1}^m \prod_{k=0}^t (A_{ik})^{j^k}
  \]
\item check $\textsf{pok}(\textsf{public\_key}(\textsf{vo}_j))$
\end{itemize}

\subsubsection{Threshold parameters}
\label{threshold-params}

The \texttt{threshold\_parameters} structure embeds data that is
published during the election.
\begin{gather*}
  \texttt{threshold\_parameters}=\left\{
    \begin{array}{rcl}
      \textsf{threshold}&:&\I\\
      \textsf{certs}&:&\texttt{cert}^\ast\\
      \textsf{coefexps}&:&\texttt{coefexps}^\ast\\
      \textsf{verification\_keys}&:&\texttt{trustee\_public\_key}^\ast
    \end{array}
  \right\}
\end{gather*}
The administrator fills it as follows:
\begin{itemize}
\item \textsf{threshold} is set to $t+1$
\item \textsf{certs} is set to $\Gamma=\gamma_1,\dotsc,\gamma_m$
\item \textsf{coefexps} is set to the same value as the
  \textsf{coefexps} field of \texttt{vinput}s
\item \textsf{verification\_keys} is set to
  $\textsf{public\_key}(\textsf{vo}_1),\dotsc,\textsf{public\_key}(\textsf{vo}_m)$
\end{itemize}

\subsection{Trustees}
\label{trustees}

\begin{gather*}
  \texttt{trustees}=\texttt{trustee\_kind}^\ast\\
  \texttt{trustee\_kind}=
  [\texttt{"Single"},\texttt{trustee\_public\_key}]\mid
  [\texttt{"Pedersen"},\texttt{threshold\_parameters}]
\end{gather*}

A $\texttt{trustees}$ structure is associated to each election. Such a
structure is a list of either a single verification key as described in
section~\ref{trustee-keys}, or threshold parameters as described in
section~\ref{threshold}.  Each item describes how a partial decryption
is computed: either a specific (mandatory) verification key is used to
compute a share, or a subset of a set of (optional) verification keys
are used to compute a share.

The generality of this definition allows to mix mandatory and optional
trustees during decryption. For example, in an election with 3
mandatory trustees, the $\texttt{trustees}$ structure will look like:
\[
  [[\texttt{"Single"},\dotsc],[\texttt{"Single"},\dotsc],[\texttt{"Single"},\dotsc]]
\]
and in an election where only one trustee is mandatory, and a subset
of another set of trustees (with a threshold) is needed to decrypt the
result, will have a $\texttt{trustees}$ structure that looks like:
\[
  [[\texttt{"Single"},\dotsc],[\texttt{"Pedersen"},\dotsc]]
\]
As explained in section~\ref{process-filling-trustees}, the sub-keys
of each item (\texttt{"Single"} or \texttt{"Pedersen"}) are then
combined to form the global election key.

\subsection{Credentials}
\label{credentials}

\newcommand{\secret}{\texttt{secret}}

A secret \emph{credential} $c$ is a string of the form
$\texttt{XXX-XXX-XXX-XXX-XXX}$ or $\texttt{XXXXXXXXXXXXXXX}$, where
the 15 $\texttt{X}$ characters are taken from the set:

\[\texttt{123456789ABCDEFGHJKLMNPQRSTUVWXYZabcdefghijkmnopqrstuvwxyz}\]
The first 14 characters are random, and the last one is a checksum to
detect typing errors. To compute the checksum, each character is
interpreted as a base 58 digit: $\texttt{1}$ is $0$, $\texttt{2}$ is
$1$, \dots, $\texttt{z}$ is $57$. The first 14 characters are
interpreted as a big-endian number $c_1$ The checksum is $53-c_1\mod
53$.

From this string, a secret exponent $s=\secret(c)$ is derived by using
PBKDF2 (RFC 2898) with:
\begin{itemize}
\item $c$ as password;
\item HMAC-SHA256 (RFC 2104, FIPS PUB 180-2) as pseudorandom function;
\item the $\uuid$ of the election as salt;
\item $1000$ iterations
\end{itemize}
and an output size of 1 block, which is interpreted as a big-endian
256-bit number and then reduced modulo $q$ to form $s$.  From this
secret exponent, a public key $\public(c)=g^s$ is computed.

\subsection{Questions}
\label{questions}

\newcommand{\question}{\texttt{question}}

\newcommand{\blank}{\textsf{blank}}
\newcommand{\minlabel}{\textsf{min}}
\newcommand{\maxlabel}{\textsf{max}}
\newcommand{\answers}{\textsf{answers}}

\begin{gather*}
  \texttt{question\_h}=\left\{
    \begin{array}{rcl}
      \answers&:&\jstring^\ast\\
      ?\blank&:&\B\\
      \minlabel&:&\I\\
      \maxlabel&:&\I\\
      \textsf{question}&:&\jstring
    \end{array}
  \right\}
  \qquad
  \begin{array}{ccc}
    \texttt{question\_nh}&=&\left\{
      \begin{array}{rcl}
        \answers&:&\jstring^\ast\\
        \textsf{question}&:&\jstring
      \end{array}
    \right\}\\
    \texttt{question\_gen}&=&\left\{
      \begin{array}{rcl}
        \textsf{type}&:&\jstring\\
        \textsf{value}&:&\texttt{json}
      \end{array}
    \right\}
  \end{array}
\end{gather*}

\begin{gather*}
  \question=\texttt{question\_h}\mid\texttt{question\_gen}
\end{gather*}

There are two types of questions: homomorphic ones and non-homomorphic
ones. The difference is in the outcome of the election: with a
homomorphic question, only the pointwise sum of all the answers
(see~\ref{answers}) will be revealed at the end of the election
whereas with a non-homomorphic question, each individual answer will
be revealed.

\subsubsection{Homomorphic questions}

Homomorphic questions are represented directly (first
alternative). They are the first type of question that was implemented
in Belenios. They are suitable for many elections, like the ones where
the voter is invited to select one choice among several (as in a
referendum).

The $\blank$ field of $\texttt{question\_h}$ is optional. When present
and true, the voter can vote blank for this question. In a blank vote,
all answers are set to $0$ regardless of the values of $\minlabel$ and
$\maxlabel$ ($\minlabel$ doesn't need to be $0$).

\subsubsection{Non-homomorphic questions}

Non-homomorphic questions are represented nested in a
\texttt{question\_gen} structure (second alternative), where the
\textsf{type} property is set to \verb=NonHomomorphic=, and the
\textsf{value} property is set to a \texttt{question\_nh}
structure. They are needed when homomorphic questions are not
suitable, for example when answers represent preferences or are too
big.

\subsection{Elections}
\label{elections}

\newcommand{\answer}{\texttt{answer}}
\newcommand{\signature}{\texttt{signature}}
\newcommand{\iproofs}{\textsf{individual\_proofs}}
\newcommand{\oproof}{\textsf{overall\_proof}}
\newcommand{\bproof}{\textsf{blank\_proof}}
\newcommand{\choices}{\textsf{choices}}
\newcommand{\iprove}{\textsf{iprove}}

\begin{gather*}
  \election=\left\{
    \begin{array}{rcl}
      \textsf{version}&:&\I\\
      \textsf{description}&:&\jstring\\
      \textsf{name}&:&\jstring\\
      \textsf{group}&:&\jstring\\
      \textsf{public\_key}&:&\G\\
      \textsf{questions}&:&\texttt{question}^\ast\\
      \textsf{uuid}&:&\texttt{uuid}\\
      \textsf{?administrator}&:&\jstring\\
      \textsf{?credential\_authority}&:&\jstring
    \end{array}
  \right\}
\end{gather*}

The $\election$ structure includes all public data related to an
election and is sent to each voter, serialized as a string which must
be always the same throughout the election. The $\textsf{version}$ is
set to 1 in this version of the specification. The group is specified
by the $\textsf{group}$ member, either $\texttt{BELENIOS-2048}$ or
$\texttt{RFC-3526-2048}$. These groups are described in
section~\ref{default-group}, using the following structures:
\begin{gather*}
  \texttt{embedding}=\left\{
    \begin{array}{rcl}
      \textsf{padding}&:&\I\\
      \textsf{bits\_per\_int}&:&\I
    \end{array}
  \right\}
  \qquad
  \texttt{group}=\left\{
    \begin{array}{rcl}
      \textsf{g}&:&\G\\
      \textsf{p}&:&\N\\
      \textsf{q}&:&\N\\
      \textsf{?embedding}&:&\texttt{embedding}
    \end{array}
  \right\}
\end{gather*}
The election public key, which is denoted by $y$ thoughout this
document, is computed during the setup phase, and stored in the
$\textsf{public\_key}$ member. The \texttt{embedding} structure is
required when the election includes a non-homomorphic question; its
meaning will be explained in section~\ref{nh-answers}.

During an election, the following data need to be public in order to
verify the setup phase and to validate ballots:
\begin{itemize}
\item the serialization of the $\election$ structure described above;
\item the $\texttt{trustees}$ structure described in
  section~\ref{trustees};
\item the set $L$ of public credentials.
\end{itemize}

\subsection{Encrypted answers}
\label{answers}

\begin{gather*}
  \texttt{answer\_h}=\left\{
    \begin{array}{rcl}
      \choices&:&\ciphertext^\ast\\
      \iproofs&:&\iproof^\ast\\
      \oproof&:&\iproof\\
      ?\bproof&:&\proof^2
    \end{array}
  \right\}
\end{gather*}

\begin{gather*}
  \texttt{answer\_nh}=\left\{
    \begin{array}{rcl}
      \choices&:&\ciphertext\\
      \textsf{proof}&:&\proof
    \end{array}
  \right\}
\end{gather*}

\begin{gather*}
  \answer=\texttt{answer\_h}\mid\texttt{answer\_nh}
\end{gather*}

The structure of an answer to a \hyperref[elections]{$\question$}
depends on the type of the question. In all cases, a
\hyperref[credentials]{credential} $c$ is needed. Let $s=\secret(c)$,
and $S=g^s$ written in base 10.

\subsubsection{Homomorphic answers}

An answer to a homomorphic question is the vector
$\choices$ of encrypted values given to each answer. When $\blank$ is
false (or absent), a blank vote is not allowed and this vector has the
same length as $\answers$; otherwise, a blank vote is allowed and this
vector has an additionnal leading value corresponding to whether the
vote is blank or not.  Each value comes with a proof (in \iproofs,
same length as \choices) that it is $0$ or $1$. The whole answer also
comes with additional proofs that values respect constraints.

More concretely, each value $m\in[0\dots1]$ is encrypted (in an El
Gamal-like fashion) into a $\ciphertext$ as follows:
\begin{enumerate}
\item pick a random $r\in\Z_q$
\item $\alphalabel=g^r$
\item $\betalabel=y^rg^m$
\end{enumerate}
where $y$ is the election public key.

The individual proof that $m\in[0\dots1]$ is computed by running
$\iprove(S,r,m,0,1)$ (see section~\ref{iproof}).

When a blank vote is not allowed, $\oproof$ proves that
$M\in[\minlabel\dots\maxlabel]$ and is computed by running
$\iprove(S,R,M-\minlabel,\minlabel,\dots,\maxlabel)$ where $R$ is the
sum of the $r$ used in ciphertexts, and $M$ the sum of the $m$. There
is no $\bproof$.

When a blank vote is allowed, and there are $n$ choices, the answer is
modeled as a vector $(m_0,m_1,\dotsc,m_n)$, when $m_0$ is whether this
is a blank vote or not, and $m_i$ (for $i>0$) is whether choice $i$
has been selected. Each $m_i$ is encrypted and proven equal to $0$ or
$1$ as above. Let $m_\Sigma=m_1+\dotsb+m_n$. The additionnal proofs
are as follows:
\begin{itemize}
\item $\bproof$ proves that $m_0=0\lor m_\Sigma=0$;
\item $\oproof$ proves that $m_0=1\lor m_\Sigma\in[\minlabel\dots\maxlabel]$.
\end{itemize}
They are computed as described in section~\ref{bproof}.

\subsubsection{Non-homomorphic answers}
\label{nh-answers}

The plaintext answer to a non-homomorphic question is a vector
$[v_1,\dots,v_n]$ of small integers, one for each possible
choice. When an election contains such a question, its \texttt{group}
structure must include an \textsf{embedding} field, specifying how the
vector of integers will be encoded into a single ciphertext:
\begin{itemize}
\item in the following, \textsf{bits\_per\_int} is denoted by $\kappa$
  and \textsf{padding} by $p$;
\item it is assumed that each $v_i$ is $\kappa$ bits (or less);
\item $[v_1,\dots,v_n]$ is encoded as:
  \[
    \xi=\textsf{group\_encode}_{\kappa,p}([v_1,\dots,v_n])=(((v_1\times
    2^\kappa+v_2)\times 2^\kappa+\dotsb)\times 2^\kappa+v_n)\times
    2^p+\varepsilon
  \]
  where $\varepsilon$ (of $p$ bits or less) is chosen so that $\xi\in\G$;
\item $\textsf{choices}$ is set to an El Gamal encryption of $\xi$ as
  follows:
  \begin{enumerate}
  \item pick a random $r\in\Z_q$
  \item $\alphalabel=g^r$
  \item $\betalabel=y^r\xi$
  \end{enumerate}
  where $y$ is the election public key;
\item $\textsf{proof}$ is computed as follows:
  \begin{enumerate}
  \item pick a random $w\in\Z_q$
  \item compute $A=g^w$
  \item $\challenge=\Hash_{\textsf{raweg}}(S,y,\alphalabel,\betalabel,A)$
  \item $\response=w-r\times\challenge$
  \end{enumerate}
  where $\Hash_{\textsf{raweg}}$ is computed as follows:
  \[
    \Hash_{\textsf{raweg}}(S,y,\alpha,\beta,A)=\shatwo(\verb=raweg|=S\verb=|=y\verb=,=\alpha\verb=,=\beta\verb=|=A)\mod q
  \]
  where \verb=raweg=, the vertical bars and the commas are verbatim and
  numbers are written in base 10. The result is interpreted as a 256-bit
  big-endian number.
\end{itemize}
The proof is verified as follows:
\begin{enumerate}
\item compute $A=g^\response\times\alphalabel^\challenge$
\item check that $\challenge=\Hash_{\textsf{raweg}}(S,y,\alphalabel,\betalabel,A)$
\end{enumerate}

\subsection{Proofs of interval membership}
\label{iproof}

\begin{gather*}
  \iproof=\proof^\ast
\end{gather*}

Given a pair $(\alpha,\beta)$ of group elements, one can prove that it
has the form $(g^r,y^rg^{M_i})$ with $M_i\in[M_0,\dots,M_k]$ by
creating a sequence of $\proof$s $\pi_0,\dots,\pi_k$ with the
following procedure, parameterised by a group element $S$:
\begin{enumerate}
\item for $j\neq i$:
  \begin{enumerate}
  \item create $\pi_j$ with a random $\challenge$ and a random
    $\response$
  \item compute
    \[A_j=\frac{g^\response}{\alpha^\challenge}\quad\text{and}\quad
    B_j=\frac{y^\response}{(\beta/g^{M_j})^\challenge}\]
  \end{enumerate}
\item $\pi_i$ is created as follows:
  \begin{enumerate}
  \item pick a random $w\in\Z_q$
  \item compute $A_i=g^w$ and $B_i=y^w$
  \item $\challenge(\pi_i)=\Hash_\iprove(S,\alpha,\beta,A_0,B_0,\dots,A_k,B_k)-\sum_{j\neq
      i}\challenge(\pi_j)\mod q$
  \item $\response(\pi_i)=w+r\times\challenge(\pi_i)\mod q$
  \end{enumerate}
\end{enumerate}
In the above, $\Hash_\iprove$ is computed as follows:
\[\Hash_\iprove(S,\alpha,\beta,A_0,B_0,\dots,A_k,B_k)=\shatwo(\verb=prove|=S\verb=|=\alpha\verb=,=\beta\verb=|=A_0\verb=,=B_0\verb=,=\dots\verb=,=A_k\verb=,=B_k)\mod q\]
where \verb=prove=, the vertical bars and the commas are verbatim and
numbers are written in base 10. The result is interpreted as a 256-bit
big-endian number. We will denote the whole procedure by
$\iprove(S,r,i,M_0,\dots,M_k)$.

The proof is verified as follows:
\begin{enumerate}
\item for $j\in[0\dots k]$, compute
  \[A_j=\frac{g^{\response(\pi_j)}}{\alpha^{\challenge(\pi_j)}}\quad\text{and}\quad
  B_j=\frac{y^{\response(\pi_j)}}{(\beta/g^{M_j})^{\challenge(\pi_j)}}\]
\item check that
  \[\Hash_\iprove(S,\alpha,\beta,A_0,B_0,\dots,A_k,B_k)=\sum_{j\in[0\dots
    k]}\challenge(\pi_j)\mod q\]
\end{enumerate}

\subsection{Proofs of possibly-blank votes}
\label{bproof}

In this section, we suppose:
\[
(\alpha_0,\beta_0)=(g^{r_0},y^{r_0}g^{m_0})
\quad\text{and}\quad
(\alpha_\Sigma,\beta_\Sigma)=(g^{r_\Sigma},y^{r_\Sigma}g^{m_\Sigma})
\]
Note that $\alpha_\Sigma$, $\beta_\Sigma$ and $r_\Sigma$ can be easily
computed from the encryptions of $m_1,\dotsc,m_n$ and their associated
secrets.

Additionnally, let $P$ be the string
``$g\verb=,=y\verb=,=\alpha_0\verb=,=\beta_0\verb=,=\alpha_\Sigma\verb=,=\beta_\Sigma$'',
where the commas are verbatim and the numbers are written in base
10. Let also $M_1,\dotsc,M_k$ be the sequence
$\minlabel,\dots,\maxlabel$ ($k=\maxlabel-\minlabel+1$).

\subsubsection{Non-blank votes ($m_0=0$)}
\label{non-blank-votes}

\paragraph{Computing \bproof}
In $m_0=0\lor m_\Sigma=0$, the first case is true. The proof $\bproof$
of the whole statement is the couple of proofs $(\pi_0,\pi_\Sigma)$
built as follows:
\begin{enumerate}
\item pick random $\challenge(\pi_\Sigma)$ and $\response(\pi_\Sigma)$
  in $\Z_q$
\item compute
  $A_\Sigma=g^{\response(\pi_\Sigma)}\times\alpha_\Sigma^{\challenge(\pi_\Sigma)}$
  and
  $B_\Sigma=y^{\response(\pi_\Sigma)}\times\beta_\Sigma^{\challenge(\pi_\Sigma)}$
\item pick a random $w$ in $\Z_q$
\item compute $A_0=g^w$ and $B_0=y^w$
\item compute \[\challenge(\pi_0)=\Hash_{\mathsf{bproof0}}(S,P,A_0,B_0,A_\Sigma,B_\Sigma)-\challenge(\pi_\Sigma)\mod q\]
\item compute $\response(\pi_0)=w-r_0\times\challenge(\pi_0)\mod q$
\end{enumerate}
In the above, $\Hash_{\mathsf{bproof0}}$ is computed as follows:
\[\Hash_{\mathsf{bproof0}}(\dotsc)=
\shatwo(\verb=bproof0|=S\verb=|=P\verb=|=A_0\verb=,=B_0\verb=,=A_\Sigma\verb=,=B_\Sigma)\mod q\]
where \verb=bproof0=, the vertical bars and the commas are verbatim and
numbers are written in base 10. The result is interpreted as a 256-bit
big-endian number.

\paragraph{Computing \oproof}
In $m_0=1\lor m_\Sigma\in[M_1\dots M_k]$, the second case
is true. Let $i$ be such that $m_\Sigma=M_i$. The proof of the whole
statement is a $(k+1)$-tuple $(\pi_0,\pi_1,\dotsc,\pi_k)$ built as
follows:
\begin{enumerate}
\item pick random $\challenge(\pi_0)$ and $\response(\pi_0)$
  in $\Z_q$
\item compute
  $A_0=g^{\response(\pi_0)}\times\alpha_0^{\challenge(\pi_0)}$
  and
  $B_0=y^{\response(\pi_0)}\times(\beta_0/g)^{\challenge(\pi_0)}$
\item for $j>0$ and $j\neq i$:
  \begin{enumerate}
  \item create $\pi_j$ with a random $\challenge$ and a random
    $\response$ in $\Z_q$
  \item compute
    $A_j={g^\response}\times{\alpha_\Sigma^\challenge}$ and
    $B_j={y^\response}\times{(\beta_\Sigma/g^{M_j})^\challenge}$
  \end{enumerate}
\item pick a random $w\in\Z_q$
\item compute $A_i=g^w$ and $B_i=y^w$
\item compute
  \[\challenge(\pi_i)=\Hash_{\textsf{bproof1}}(S,P,A_0,B_0,\dots,A_k,B_k)-\sum_{j\neq i}\challenge(\pi_j)\mod q\]
\item compute $\response(\pi_i)=w-r_\Sigma\times\challenge(\pi_i)\mod q$
\end{enumerate}
In the above, $\Hash_{\mathsf{bproof1}}$ is computed as follows:
\[\Hash_{\mathsf{bproof1}}(\dotsc)=
\shatwo(\verb=bproof1|=S\verb=|=P\verb=|=A_0\verb=,=B_0\verb=,=\dotsc\verb=,=A_k\verb=,=B_k)\mod q\]
where \verb=bproof1=, the vertical bars and the commas are verbatim and
numbers are written in base 10. The result is interpreted as a 256-bit
big-endian number.

\subsubsection{Blank votes ($m_0=1$)}

\paragraph{Computing \bproof}
In $m_0=0\lor m_\Sigma=0$, the second case is true. The proof
$\bproof$ of the whole statement is the couple of proofs
$(\pi_0,\pi_\Sigma)$ built as in section~\ref{non-blank-votes}, but
exchanging subscripts $0$ and $\Sigma$ everywhere except in the call
to $\Hash_{\textsf{bproof0}}$.

\paragraph{Computing \oproof}
In $m_0=1\lor m_\Sigma\in[M_1\dots M_k]$, the first case is
true. The proof of the whole statement is a $(k+1)$-tuple
$(\pi_0,\pi_1,\dotsc,\pi_k)$ built as follows:
\begin{enumerate}
\item for $j>0$:
  \begin{enumerate}
  \item create $\pi_j$ with a random $\challenge$ and a random
    $\response$ in $\Z_q$
  \item compute
    $A_j={g^\response}\times{\alpha_\Sigma^\challenge}$ and
    $B_j={y^\response}\times{(\beta_\Sigma/g^{M_j})^\challenge}$
  \end{enumerate}
\item pick a random $w\in\Z_q$
\item compute $A_0=g^w$ and $B_0=y^w$
\item compute
  \[\challenge(\pi_0)=\Hash_{\textsf{bproof1}}(S,P,A_0,B_0,\dots,A_k,B_k)-\sum_{j>0}\challenge(\pi_j)\mod q\]
\item compute $\response(\pi_0)=w-r_0\times\challenge(\pi_0)\mod q$
\end{enumerate}

\subsubsection{Verifying proofs}

\paragraph{Verifying \bproof}
A proof of $m_0=0\lor m_\Sigma=0$ is a couple of proofs
$(\pi_0,\pi_\Sigma)$ such that the following procedure passes:
\begin{enumerate}
\item compute
  $A_0=g^{\response(\pi_0)}\times\alpha_0^{\challenge(\pi_0)}$
  and
  $B_0=y^{\response(\pi_0)}\times\beta_0^{\challenge(\pi_0)}$
\item compute
  $A_\Sigma=g^{\response(\pi_\Sigma)}\times\alpha_\Sigma^{\challenge(\pi_\Sigma)}$
  and
  $B_\Sigma=y^{\response(\pi_\Sigma)}\times\beta_\Sigma^{\challenge(\pi_\Sigma)}$
\item check that
  \[\Hash_{\mathsf{bproof0}}(S,P,A_0,B_0,A_\Sigma,B_\Sigma)=\challenge(\pi_0)+\challenge(\pi_\Sigma)\mod q\]
\end{enumerate}

\paragraph{Verifying \oproof}
A proof of $m_0=1\lor m_\Sigma\in[M_1\dots M_k]$ is a $(k+1)$-tuple
$(\pi_0,\pi_1,\dotsc,\pi_k)$ such that the following procedure passes:
\begin{enumerate}
\item compute
  $A_0=g^{\response(\pi_0)}\times\alpha_0^{\challenge(\pi_0)}$
  and
  $B_0=y^{\response(\pi_0)}\times(\beta_0/g)^{\challenge(\pi_0)}$
\item for $j>0$, compute
  \[A_j=g^{\response(\pi_j)}\times\alpha_\Sigma^{\challenge(\pi_j)}
  \quad\text{and}\quad
  B_j=y^{\response(\pi_j)}\times(\beta_\Sigma/g^{M_j})^{\challenge(\pi_j)}\]
\item check that
  \[\Hash_{\textsf{bproof1}}(S,P,A_0,B_0,\dots,A_k,B_k)=\sum_{j=0}^k\challenge(\pi_j)\mod q\]
\end{enumerate}

\subsection{Signatures}
\label{signatures}

\begin{gather*}
  \signature=\left\{
    \begin{array}{rcl}
      \pklabel&:&\pk\\
      \challenge&:&\Z_q\\
      \response&:&\Z_q
    \end{array}
  \right\}
\end{gather*}

\newcommand{\siglabel}{\textsf{signature}}

Each ballot contains a (Schnorr-like) digital signature to avoid
ballot stuffing. The signature needs a
\hyperref[credentials]{credential} $c$ and uses all the \ciphertext{}s
$\gamma_1,\dots,\gamma_l$ that appear in the ballot. It is computed as
follows:
\begin{enumerate}
\item compute $s=\secret(c)$
\item pick a random $w\in\Z_q$
\item compute $A=g^w$
\item $\pklabel=g^s$
\item $\challenge=\Hash_\siglabel(\pklabel,A,\gamma_1,\dots,\gamma_l)\mod q$
\item $\response=w-s\times\challenge\mod q$
\end{enumerate}
In the above, $\Hash_\siglabel$ is computed as follows:
\[
\Hash_\siglabel(S,A,\gamma_1,\dots,\gamma_l)=\shatwo(\verb=sig|=S\verb=|=A\verb=|=\alphalabel(\gamma_1)\verb=,=\betalabel(\gamma_1)\verb=,=\dots\verb=,=\alphalabel(\gamma_l)\verb=,=\betalabel(\gamma_l))
\]
where \verb=sig=, the vertical bars and commas are verbatim and
numbers are written in base 10. The result is interpreted as a 256-bit
big-endian number.

Signatures are verified as follows:
\begin{enumerate}
\item compute $A=g^\response\times \pklabel^\challenge$
\item check that $\challenge=\Hash_\siglabel(\pklabel,A,\gamma_1,\dots,\gamma_l)\mod q$
\end{enumerate}

\subsection{Ballots}
\label{ballots}

\newcommand{\json}{\textsf{JSON}}

\begin{gather*}
  \ballot=\left\{
    \begin{array}{rcl}
      \answers&:&\hyperref[answers]{\answer}^\ast\\
      \textsf{election\_hash}&:&\jstring\\
      \textsf{election\_uuid}&:&\uuid\\
      \siglabel&:&\hyperref[signatures]{\signature}
    \end{array}
  \right\}
\end{gather*}
The so-called hash (or \emph{fingerprint}) of the election
is computed with the function $\Hash_\json$:
\[
\Hash_\json(J)=\basesixfour(\shatwo(J))
\]
Where $J$ is the serialization of the $\election$ structure, and the
Base64 encoding is done without padding.

The same hashing function is used on a serialization (done by the
voting client) of the $\ballot$ structure to produce a so-called
\emph{smart ballot tracker}.

The weight of a ballot $B$, denoted by $\textsf{weight}(B)$, is the
weight associated to $\pklabel(\siglabel(B))$ in the list of public
credentials $L$.

\subsection{Encrypted tally}
\label{tally}

\begin{gather*}
  \texttt{ciphertexts\_h}=\ciphertext^\ast
  \qquad
  \texttt{ciphertexts\_nh}=\ciphertext^\ast
  \\
  \etally=(\texttt{ciphertexts\_h}\mid\texttt{ciphertexts\_nh})^\ast
\end{gather*}
A so-called \emph{encrypted tally} is constructed out of the accepted
ballots $B_1,\dots,B_n$.  It is an array $[C_1,\dots,C_m]$ where $m$
is the number of questions. Each element $C_i$ is itself an array of
ciphertexts that is built differently depending on the type of the
question:
\begin{itemize}
\item for homomorphic questions, each element of $C_i$
  ($\texttt{ciphertexts\_h}$) is the pointwise product of the $i$-th
  ciphertext of all the ballots, raised to the power of its weight:
  \[
    C_{i,j}=\prod_{k}\choices(\answers(B_k)_i)_j^{\textsf{weight}(B_k)}
  \]
  where the product of two ciphertexts $(\alpha_1,\beta_1)$ and
  $(\alpha_2,\beta_2)$ is $(\alpha_1\alpha_2,\beta_1\beta_2)$;
\item for non-homomorphic questions, $C_i$ is directly made from the
  list of ciphertexts corresponding to the question:
  \[
    C_{i,k}=\choices(\answers(B_k)_i)
  \]
  In this case, it is an error if $\textsf{weight}(B_k)\neq 1$.
\end{itemize}
In the end, in both cases, the encrypted tally is isomorphic to an
array of arrays of ciphertexts:
\[
\etally\approx\ciphertext^\ast{}^\ast
\]

\subsection{Shuffles}
\label{shuffles}

If the election has non-homomorphic questions, let us say $n$ out of
$m$ ($1\leq n\leq m$), non-homomorphic ciphertexts must be
shuffled. They are first extracted from the encrypted tally~$a$: if
$i_1,\dots,i_n$ are the indices of the non-homomorphic questions,
\[
  b=\textsf{nh\_ciphertexts}(a)=[a_{i_1},\dots,a_{i_n}]
\]
where $a$ is the $\etally$ structure defined
in~\ref{tally}. Conversely, once ciphertexts are shuffled as $b'$ (see
later), they must be merged into the encrypted tally as $a'$ such that
$b'=\textsf{nh\_ciphertexts}(a')$.

Shuffles are done in the same way as the CHVote system\footnote{See
  version 1.3.2 of the CHVote System Specification at~\cite{CHVote}}.
%  \url{https://eprint.iacr.org/2017/325}}. 
For each non-homomorphic
question, its ciphertexts are re-encrypted and randomly
permuted, and a zero-knowledge proof of the permutation is
computed. All these shuffles are then assembled into a
$\texttt{shuffle}$ structure:
\begin{gather*}
  \texttt{shuffle}=\left\{
    \begin{array}{rcl}
      \textsf{ciphertexts}&:&\ciphertext^\ast{}^\ast\\
      \textsf{proofs}&:&\texttt{shuffle\_proof}^\ast
    \end{array}
  \right\}
\end{gather*}
which uses the following auxiliary types:
\begin{gather*}
  \begin{array}{rcl}
    \texttt{shuffle\_commitment\_rand}&=&\G\times\G\times\G\times(\G\times\G)\times\G^\ast\\
    \texttt{shuffle\_response}&=&\Z_q\times\Z_q\times\Z_q\times\Z_q\times\Z_q^\ast\times\Z_q^\ast\\
    \texttt{shuffle\_commitment\_perm}&=&\G^\ast\\
    \texttt{shuffle\_chained\_challenges}&=&\G^\ast\\
    \texttt{shuffle\_proof}&=&\texttt{shuffle\_commitment\_rand}\\
                                      &\times&\texttt{shuffle\_response}\\
                                      &\times&\texttt{shuffle\_commitment\_perm}\\
                                      &\times&\texttt{shuffle\_chained\_challenges}
  \end{array}
\end{gather*}
For each non-homomorphic question $i$:
\begin{enumerate}
\item let $\textbf{e}=b_i=[e_1,\dots,e_N]$ be the array of ciphertexts
  corresponding to question $i$ ($N$ being the number of ballots);
\item let
  $(\textbf{e}',\textbf{r}',\psi)=\textsf{GenShuffle}(\textbf{e},y)$
  ($y$ being the public key of the election);
\item let
  $\pi=\textsf{GenShuffleProof}(\textbf{e},\textbf{e}',\textbf{r}',\psi,y)$;
  \item set $\textsf{ciphertexts}_i$ to $\textbf{e}'$ and
    $\textsf{proofs}_i$ to $\pi$.
\end{enumerate}
The functions $\textsf{GenShuffle}$ and $\textsf{GenShuffleProof}$ are
the same as in CHVote and are given in
section~\ref{shuffle-algorithms}. Typically, several shuffles will be
computed sequentially by different persons.

\subsection{Partial decryptions}

\newcommand{\dfactors}{\textsf{decryption\_factors}}
\newcommand{\dproofs}{\textsf{decryption\_proofs}}
\newcommand{\decrypt}{\textsf{decrypt}}

\begin{gather*}
  \pdecryption=\left\{
    \begin{array}{rcl}
      \dfactors&:&\G^\ast{}^\ast\\
      \dproofs&:&\proof^\ast{}^\ast
    \end{array}
  \right\}
\end{gather*}
From the encrypted tally $a'$ (where answers to non-homomorphic questions
have been shuffled), each trustee computes a partial decryption using
the \hyperref[trustee-keys]{private key} $x$ (and the corresponding
public key $X=g^x$) he generated during election setup. It consists of
so-called \emph{decryption factors}:
\[
\dfactors_{i,j}=\alphalabel(a'_{i,j})^x
\]
and proofs that they were correctly computed. Each $\dproofs_{i,j}$ is
computed as follows:
\begin{enumerate}
\item pick a random $w\in\Z_q$
\item compute $A=g^w$ and $B=\alphalabel(a'_{i,j})^w$
\item $\challenge=\Hash_\decrypt(X,A,B)$
\item $\response=w+x\times\challenge\mod q$
\end{enumerate}
In the above, $\Hash_\decrypt$ is computed as follows:
\[
\Hash_\decrypt(X,A,B)=\shatwo(\verb=decrypt|=X\verb=|=A\verb=,=B)\mod q
\]
where \verb=decrypt=, the vertical bars and the comma are verbatim and
numbers are written in base 10. The result is interpreted as a 256-bit
big-endian number.

These proofs are verified using the $\tpk$ structure $k$ that the
trustee sent to the administrator during the election setup:
\begin{enumerate}
\item compute
\[
A=\frac{g^\response}{\pklabel(k)^\challenge}
\quad\text{and}\quad
B=\frac{\alphalabel(a'_{i,j})^\response}{\dfactors_{i,j}^\challenge}
\]
\item check that $\Hash_\decrypt(\pklabel(k),A,B)=\challenge$
\end{enumerate}

\subsection{Election result}
\label{election-result}

\newcommand{\ntallied}{\textsf{num\_tallied}}
\newcommand{\etallylabel}{\textsf{encrypted\_tally}}
\newcommand{\pdlabel}{\textsf{partial\_decryptions}}
\newcommand{\resultlabel}{\textsf{result}}

\begin{gather*}
  \result=\left\{
    \begin{array}{rcl}
      \ntallied&:&\I\\
      \etallylabel&:&\etally\\
      \textsf{?shuffles}&:&\texttt{shuffle}^\ast\\
      \pdlabel&:&\pdecryption^\ast\\
      \resultlabel&:&(\I^\ast\mid\I^\ast{}^\ast)^\ast
    \end{array}
  \right\}
\end{gather*}
The $\etallylabel$ field is set to the encrypted tally $a'$.

The decryption factors are combined for each ciphertext to build
synthetic ones $F_{i,j}$. The way this combination is done depends on
the $\texttt{trustees}$ structure, the list $PK$. For each item of
index $\tau$ in $PK$, a sub-factor $F_{i,j,\tau}$ is computed:
\begin{itemize}
\item for a \texttt{"Single"} item corresponding to trustee
  $\mathcal{T}_z$:
  \[
    F_{i,j,\tau}=\pdlabel_{z,i,j}
  \]
\item for a \texttt{"Pedersen"} item corresponding to trustees
  $\mathcal{T}_{z_1},\dots,\mathcal{T}_{z_\mu}$:
  \[
    F_{i,j,\tau}=\prod_{\delta\in\mathcal{I}}(\pdlabel_{z_\delta,i,j})^{\lambda_\delta^{\mathcal{I}}}
  \]
  where $\mathcal{I}$ is the set of ($t+1$) indexes of supplied
  partial decryptions, relative to
  $\mathcal{T}_{z_1},\dots,\mathcal{T}_{z_\mu}$ (i.e. $\mathcal{I}\subseteq\{1,\dots,\mu\}$), and
  $\lambda_\delta^{\mathcal{I}}$ are the Lagrange coefficients:
  \[
    \lambda_\delta^{\mathcal{I}}=\prod_{k\in\mathcal{I}\backslash\{\delta\}}\frac{k}{k-\delta}\mod q
  \]
\end{itemize}
The synthetic factor is then computed as the product of all sub-factors:
\[
  F_{i,j}=\prod_\tau F_{i,j,\tau}
\]

The $\resultlabel$ field of the $\result$ structure is then computed
as follows:
\begin{itemize}
\item if question $i$ is homomorphic,
  \[
    \resultlabel_{i,j}=\log_g\left(\frac{\betalabel(a'_{i,j})}{F_{i,j}}\right)
  \]
  where $j$ represents an answer. The discrete logarithm can be easily
  computed because it is bounded by the sum of all weights;
\item if question $i$ is non-homomorphic,
  \[
    \resultlabel_{i,j}=\textsf{group\_decode}_{\kappa,p}\left(\frac{\betalabel(a'_{i,j})}{F_{i,j}}\right)
  \]
  where $j$ represents a ballot, and $\textsf{group\_decode}$ is the
  inverse of $\textsf{group\_encode}$ from section~\ref{nh-answers}.
\end{itemize}
If the election has non-homomorphic questions, the $\textsf{shuffles}$
field is set to the computed $\texttt{shuffle}$ structures; otherwise,
it is absent.

After the election, the following data needs to be public in order to
verify the tally:
\begin{itemize}
\item the $\election$ structure;
\item all the $\tpk$s, or the $\texttt{threshold\_parameters}$, that
  were generated during the \hyperref[election-setup]{setup phase};
\item the set of public credentials;
\item the set of ballots;
\item the $\result$ structure described above.
\end{itemize}

\section{Group parameters}
\label{default-group}

\subsection{BELENIOS-2048}
\label{h-group}

This group is optimized for elections that have only homomorphic
questions and is used in this case. Its parameters have been generated
by the \verb=fips.sage= script (available in Belenios sources), which
is itself based on FIPS 186-4.

\[
\begin{array}{lcr}
\textsf{p}&=&20694785691422546\\
&&401013643657505008064922989295751104097100884787057374219242\\
&&717401922237254497684338129066633138078958404960054389636289\\
&&796393038773905722803605973749427671376777618898589872735865\\
&&049081167099310535867780980030790491654063777173764198678527\\
&&273474476341835600035698305193144284561701911000786737307333\\
&&564123971732897913240474578834468260652327974647951137672658\\
&&693582180046317922073668860052627186363386088796882120769432\\
&&366149491002923444346373222145884100586421050242120365433561\\
&&201320481118852408731077014151666200162313177169372189248078\\
&&507711827842317498073276598828825169183103125680162072880719\\
\textsf{g}&=&2402352677501852\\
&&209227687703532399932712287657378364916510075318787663274146\\
&&353219320285676155269678799694668298749389095083896573425601\\
&&900601068477164491735474137283104610458681314511781646755400\\
&&527402889846139864532661215055797097162016168270312886432456\\
&&663834863635782106154918419982534315189740658186868651151358\\
&&576410138882215396016043228843603930989333662772848406593138\\
&&406010231675095763777982665103606822406635076697764025346253\\
&&773085133173495194248967754052573659049492477631475991575198\\
&&775177711481490920456600205478127054728238140972518639858334\\
&&115700568353695553423781475582491896050296680037745308460627\\
\textsf{q}&=&78571733251071885\\
&&079927659812671450121821421258408794611510081919805623223441
\end{array}
\]

The additional output of the generation algorithm is:
\[
\begin{array}{lcr}
\textsf{domain\_parameter\_seed}&=&478953892617249466\\
&&166106476098847626563138168027\\
&&716882488732447198349000396592\\
&&020632875172724552145560167746\\
\textsf{counter}&=&109
\end{array}
\]

\subsection{RFC-3526-2048}
\label{nh-group}

The group described in the previous section is not suitable for
encoding non-homomorphic answers (the $\textsf{group\_encode}$
function of section~\ref{nh-answers}). Therefore, we use a different
group if the election has non-homomorphic questions. This group is
the 2048-bit one defined in RFC 3526:
\[
\begin{array}{lcr}
\textsf{p}&=&32317006071311007\\
&&300338913926423828248817941241140239112842009751400741706634\\
&&354222619689417363569347117901737909704191754605873209195028\\
&&853758986185622153212175412514901774520270235796078236248884\\
&&246189477587641105928646099411723245426622522193230540919037\\
&&680524235519125679715870117001058055877651038861847280257976\\
&&054903569732561526167081339361799541336476559160368317896729\\
&&073178384589680639671900977202194168647225871031411336429319\\
&&536193471636533209717077448227988588565369208645296636077250\\
&&268955505928362751121174096972998068410554359584866583291642\\
&&136218231078990999448652468262416972035911852507045361090559\\
\textsf{g}&=&2\\
\textsf{q}&=&16158503035655503\\
&&650169456963211914124408970620570119556421004875700370853317\\
&&177111309844708681784673558950868954852095877302936604597514\\
&&426879493092811076606087706257450887260135117898039118124442\\
&&123094738793820552964323049705861622713311261096615270459518\\
&&840262117759562839857935058500529027938825519430923640128988\\
&&027451784866280763083540669680899770668238279580184158948364\\
&&536589192294840319835950488601097084323612935515705668214659\\
&&768096735818266604858538724113994294282684604322648318038625\\
&&134477752964181375560587048486499034205277179792433291645821\\
&&068109115539495499724326234131208486017955926253522680545279
\end{array}
\]
Additionally, its \textsf{embedding} field is set to:
\[
  \left\{
    \begin{array}{rcl}
      \textsf{padding}&=&8\\
      \textsf{bits\_per\_int}&=&8
    \end{array}
  \right\}
\]

\section{Shuffle algorithms}
\label{shuffle-algorithms}

The algorithms $\textsf{GenShuffle}$ and $\textsf{GenShuffleProof}$
are referred to in section~\ref{shuffles}. They were taken from
version 1.3.2 of the CHVote System
Specification~\cite{CHVote},
%\footnote{\url{https://eprint.iacr.org/2017/325}}, 
and
are given here for self-completeness. We also give the
$\textsf{CheckShuffleProof}$ algorithm, used to check a proof produced
by $\textsf{GenShuffleProof}$. For more explanations on these
algorithms, please refer to the CHVote System Specification.

\begin{table}
  \begin{framed}
    \noindent\paragraph{Input}
    \begin{itemize}
    \item $\textbf{e}=[e_1,\dots,e_N]\in\texttt{ciphertext}^N$:
      encrypted answers to one non-homomorphic question
    \item $y\in\G$: public key of the election
    \end{itemize}
    \noindent\paragraph{Algorithm}
    \begin{enumerate}
    \item $\psi\leftarrow\textsf{GenPermutation}(N)$\hfill//
      $\psi=[j_1,\dots,j_N]$, see table~\ref{gen-permutation}
    \item For $i=1,\dots,N$:
      \begin{itemize}
      \item
        $(e'_i,r'_i)\leftarrow\textsf{GenReEncryption}(e_i,y)$\hfill//
        see table~\ref{gen-re-encryption}
      \end{itemize}
    \item $\textbf{e}'\leftarrow[e'_{j_1},\dots,e'_{j_N}]$
    \item $\textbf{r}'\leftarrow[r'_1,\dots,r'_N]$
    \item Return $(\textbf{e}',\textbf{r}',\psi)$\hfill//
      $\textbf{e}'\in\texttt{ciphertext}^N$, $\textbf{r}'\in\Z_q^N$,
      $\psi\in\Psi_N$
    \end{enumerate}
  \end{framed}
  \caption{Function $\mathsf{GenShuffle}(\textbf{e},y)$}
  \label{gen-shuffle}
\end{table}

\begin{table}
  \begin{framed}
    \noindent\paragraph{Input}
    \begin{itemize}
    \item $N\in\N$: permutation size
    \end{itemize}
    \noindent\paragraph{Algorithm}
    \begin{enumerate}
    \item $I\leftarrow[1,\dots,N]$
    \item For $i=0,\dots,N-1$:
      \begin{enumerate}
      \item Pick $k$ uniformly at random in $\{i,\dots,N-1\}$
      \item $j_{i+1}\leftarrow I[k]$
      \item $I[k]\leftarrow I[i]$
      \end{enumerate}
    \item $\psi\leftarrow[j_1,\dots,j_N]$
    \item Return $\psi$\hfill// $\psi\in\Psi_N$
    \end{enumerate}
  \end{framed}
  \caption{Function $\mathsf{GenPermutation}(N)$}
  \label{gen-permutation}
\end{table}

\begin{table}
  \begin{framed}
    \noindent\paragraph{Input}
    \begin{itemize}
    \item $e\in\texttt{ciphertext}$: one encrypted answer to one
      non-homomorphic question
    \item $y\in\G$: public key of the election
    \end{itemize}
    \noindent\paragraph{Algorithm}
    \begin{enumerate}
    \item Pick $r'$ uniformly at random in $\Z_q$
    \item $\alpha'\leftarrow\alphalabel(e)\times g^{r'}$
    \item $\beta'\leftarrow\betalabel(e)\times y^{r'}$
    \item Let $e'$ be a new $\texttt{ciphertext}$ with
      $\alphalabel=\alpha'$ and $\betalabel=\beta'$
    \item Return $(e',r')$\hfill// $e'\in\texttt{ciphertext}$,
      $r'\in\Z_q$
    \end{enumerate}
  \end{framed}
  \caption{Function $\mathsf{GenReEncryption}(e,y)$}
  \label{gen-re-encryption}
\end{table}

\begin{table}
  \small
  \begin{framed}
    \noindent\paragraph{Input}
    \begin{itemize}
    \item $\textbf{e}=[e_1,\dots,e_N]\in\texttt{ciphertext}^N$:
      encrypted answers to one question; we will denote by
      $\alpha_i$ and $\beta_i$ the contents of $e_i$
    \item $\textbf{e}'=[e'_1,\dots,e'_N]\in\texttt{ciphertext}^N$:
      shuffled encrypted answers; we will denote by
      $\alpha'_i$ and $\beta'_i$ the contents of $e'_i$
    \item $\textbf{r}'=[r'_1,\dots,r'_N]\in\Z_q^N$: re-encryption
      randomizations
    \item $\psi=[j_1,\dots,j_N]\in\Psi_N$: permutation
    \item $pk\in\G$: the public key of the election
    \end{itemize}
    \noindent\paragraph{Algorithm}
    \begin{enumerate}
    \item $h\leftarrow\textsf{GetSecondaryGenerator}()$,
      $\textbf{h}\leftarrow\textsf{GetGenerators}(N)$\hfill// see
      tables~\ref{get-secondary-generator} and \ref{get-generators}
    \item
      $(\textbf{c},\textbf{r})\leftarrow\textsf{GenPermutationCommitment}(\psi,\textbf{h})$\hfill//
      see table~\ref{gen-permutation-commitment}
    \item
      $\textsf{str}_c\leftarrow\llbracket\textbf{e}\rrbracket\llbracket\textbf{e}'\rrbracket\llbracket\textbf{c}\rrbracket$\hfill//
      see table~\ref{serialize-array}
    \item
      $\textbf{u}\leftarrow\textsf{GetNIZKPChallenges}(N,\verb=shuffle-challenges|=\textsf{str}_c)$\hfill//
      see table~\ref{get-nizkp-challenges}
    \item For $i=1,\dots,N$: $u'_i\leftarrow u_{j_i}$
    \item $\textbf{u}'\leftarrow[u'_1,\dots,u'_N]$
    \item
      $(\hat{\textbf{c}},\hat{\textbf{r}})\leftarrow\textsf{GenCommitmentChain}(h,\textbf{u}')$\hfill//
      see table~\ref{gen-commitment-chain}
    \item For $i=1,\dots,4$: pick $\omega_i$ at random in $\Z_q$
    \item For $i=1,\dots,N$: pick $\hat{\omega}_i$ and $\omega'_i$ at random in $\Z_q$
    \item $t_1\leftarrow g^{\omega_1}$, $t_2\leftarrow g^{\omega_2}$,
      $t_3\leftarrow g^{\omega_3}\prod_{i=1}^N h_i^{\omega'_i}$
    \item
      $(t_{4,1},t_{4,2})\leftarrow(pk^{-\omega_4}\prod_{i=1}^N(\beta'_i)^{\omega'_i},g^{-\omega_4}\prod_{i=1}^N(\alpha'_i)^{\omega'_i})$
    \item $\hat{c}_0\leftarrow h$
    \item For $i=1,\dots,N$:
      $\hat{t}_i\leftarrow
      g^{\hat{\omega}_i}\hat{c}_{i-1}^{\omega'_i}$
    \item
      $t\leftarrow(t_1,t_2,t_3,(t_{4,1},t_{4,2}),[\hat{t}_1,\dots,\hat{t}_N])$,
      $\textsf{str}_t\leftarrow\llbracket[t_1,t_2,t_3,t_{4,1},t_{4,2}]\rrbracket\llbracket[\hat{t}_1,\dots,\hat{t}_N]\rrbracket$
    \item
      $y\leftarrow(\textbf{e},\textbf{e}',\textbf{c},\hat{\textbf{c}},pk)$,
      $\textsf{str}_y\leftarrow\textsf{str}_c\llbracket\hat{\textbf{c}}\rrbracket
      pk$\hfill// $pk$ taken as a number in base 10
    \item
      $c\leftarrow\textsf{GetNIZKPChallenge}(\verb=shuffle-challenge|=\textsf{str}_t\textsf{str}_y)$\hfill//
      see table~\ref{get-nizkp-challenge}
    \item $\bar{r}\leftarrow\sum_{i=1}^N r_i\mod q$,
      $s_1\leftarrow\omega_1+c\times\bar{r}\mod q$
    \item $v_N\leftarrow 1$
    \item For $i=N-1,\dots,1$: $v_i\leftarrow u'_{i+1}v_{i+1}\mod q$
    \item $\hat{r}\leftarrow\sum_{i=1}^N \hat{r}_iv_i\mod q$, $s_2\leftarrow\omega_2+c\times\hat{r}\mod q$
    \item $\tilde{r}\leftarrow\sum_{i=1}^N r_iu_i\mod q$, $s_3\leftarrow\omega_3+c\times\tilde{r}\mod q$
    \item $r'\leftarrow\sum_{i=1}^N r'_iu_i\mod q$, $s_4\leftarrow\omega_4+c\times r'\mod q$
    \item For $i=1,\dots,N$:
      $\hat{s}_i\leftarrow\hat{\omega}_i+c\times\hat{r}_i\mod q$,
      $s'_i\leftarrow\omega'_i+c\times u'_i\mod q$
    \item
      $s\leftarrow(s_1,s_2,s_3,s_4,[\hat{s}_1,\dots,\hat{s}_N],[s'_1,\dots,s'_N])$
    \item $\pi\leftarrow(t,s,\textbf{c},\hat{\textbf{c}})$
    \item Return $\pi$\hfill// $\pi\in\texttt{shuffle\_proof}$
    \end{enumerate}
  \end{framed}
  \caption{Function $\mathsf{GenShuffleProof}(\textbf{e},\textbf{e}',\textbf{r}',\psi,pk)$}
  \label{gen-shuffle-proof}
\end{table}

\begin{table}
  \small
  \begin{framed}
    \noindent\paragraph{Input}
    \begin{itemize}
    \item $\pi\in\texttt{shuffle\_proof}$: shuffle proof
    \item $\textbf{e}=[e_1,\dots,e_N]\in\texttt{ciphertext}^N$:
      encrypted answers to one question; we will denote by
      $\alpha_i$ and $\beta_i$ the contents of $e_i$
    \item $\textbf{e}'=[e'_1,\dots,e'_N]\in\texttt{ciphertext}^N$:
      shuffled encrypted answers; we will denote by
      $\alpha'_i$ and $\beta'_i$ the contents of $e'_i$
    \item $pk\in\G$: the public key of the election
    \end{itemize}
    \noindent\paragraph{Algorithm}
    \begin{enumerate}
    \item $(t,s,\textbf{c},\hat{\textbf{c}})\leftarrow\pi$
    \item $(t_1,t_2,t_3,(t_{4,1},t_{4,2}),[\hat{t}_1,\dots,\hat{t}_N])\leftarrow t$
    \item $(s_1,s_2,s_3,s_4,[\hat{s}_1,\dots,\hat{s}_N],[s'_1,\dots,s'_N])\leftarrow s$
    \item $[c_1,\dots,c_N]\leftarrow\textbf{c}$,
      $[\hat{c}_1,\dots,\hat{c}_N]\leftarrow\hat{\textbf{c}}$
    \item $h\leftarrow\textsf{GetSecondaryGenerator}()$,
      $\textbf{h}\leftarrow\textsf{GetGenerators}(N)$\hfill// see
      tables~\ref{get-secondary-generator} and \ref{get-generators}
    \item
      $\textsf{str}_c\leftarrow\llbracket\textbf{e}\rrbracket\llbracket\textbf{e}'\rrbracket\llbracket\textbf{c}\rrbracket$\hfill//
      see table~\ref{serialize-array}
    \item
      $\textbf{u}\leftarrow\textsf{GetNIZKPChallenges}(N,\verb=shuffle-challenges|=\textsf{str}_c)$\hfill//
      see table~\ref{get-nizkp-challenges}
    \item
      $\textsf{str}_t\leftarrow\llbracket[t_1,t_2,t_3,t_{4,1},t_{4,2}]\rrbracket\llbracket[\hat{t}_1,\dots,\hat{t}_N]\rrbracket$
    \item
      $\textsf{str}_y\leftarrow\textsf{str}_c\llbracket\hat{\textbf{c}}\rrbracket
      pk$\hfill// $pk$ taken as a number in base 10
    \item
      $c\leftarrow\textsf{GetNIZKPChallenge}(\verb=shuffle-challenge|=\textsf{str}_t\textsf{str}_y)$\hfill//
      see table~\ref{get-nizkp-challenge}
    \item $\bar{c}\leftarrow\prod_{i=1}^N c_i/\prod_{i=1}^N h_i$
    \item $u\leftarrow\prod_{i=1}^N u_i\mod q$
    \item $\hat{c}_0\leftarrow h$
    \item $\hat{c}\leftarrow\hat{c}_N/h^u$
    \item $\tilde{c}\leftarrow\prod_{i=1}^N c_i^{u_i}$
    \item $(\alpha',\beta')\leftarrow(\prod_{i=1}^N \alpha_i^{u_i},\prod_{i=1}^N \beta_i^{u_i})$
    \item $t'_1\leftarrow\bar{c}^{-c}\times g^{s_1}$
    \item $t'_2\leftarrow\hat{c}^{-c}\times g^{s_2}$
    \item $t'_3\leftarrow\tilde{c}^{-c}\times g^{s_3}\prod_{i=1}^N h_i^{s'_i}$
    \item
      $(t'_{4,1},t'_{4,2})\leftarrow((\beta')^{-c}\times
      pk^{-s_4}\prod_{i=1}^N(\beta'_i)^{s'_i},(\alpha')^{-c}\times
      g^{-s_4}\prod_{i=1}^N(\alpha'_i)^{s'_i})$
    \item For $i=1,\dots,N$: $\hat{t'_i}\leftarrow\hat{c}_i^{-c}\times g^{\hat{s}_i}\times\hat{c}_{i-1}^{s'_i}$
    \item Return
      $(t_1=t'_1)\land(t_2=t'_2)\land(t_3=t'_3)\land(t_{4,1}=t'_{4,1})\land(t_{4,2}=t'_{4,2})\land\left[\bigwedge_{i=1}^N(\hat{t_i}=\hat{t'_i})\right]$
    \end{enumerate}
  \end{framed}
  \caption{Function $\mathsf{CheckShuffleProof}(\pi,\textbf{e},\textbf{e}',pk)$}
  \label{check-shuffle-proof}
\end{table}

\begin{table}
  \begin{framed}
    \noindent\paragraph{Algorithm}
    \begin{enumerate}
    \item $h\leftarrow\textsf{GetGenerator}(-1)$\hfill// see
      table~\ref{get-generator}
    \item Return $h$\hfill// $h\in\G^N$
    \end{enumerate}
  \end{framed}
  \caption{Function $\mathsf{GetSecondaryGenerator}()$}
  \label{get-secondary-generator}
\end{table}

\begin{table}
  \begin{framed}
    \noindent\paragraph{Input}
    \begin{itemize}
    \item $N\in\N$: number of independent generators to get
    \end{itemize}
    \noindent\paragraph{Algorithm}
    \begin{enumerate}
    \item For $i=0,\dots,N-1$:
      $h_i\leftarrow\textsf{GetGenerator}(i)$\hfill// see
      table~\ref{get-generator}
    \item $\textbf{h}\leftarrow[h_0,\dots,h_{N-1}]$
    \item Return $\textbf{h}$\hfill// $\textbf{h}\in\G^N$
    \end{enumerate}
  \end{framed}
  \caption{Function $\mathsf{GetGenerators}(N)$}
  \label{get-generators}
\end{table}

\begin{table}
  \begin{framed}
    \noindent\paragraph{Input}
    \begin{itemize}
    \item $i\in\Z$: number of the independent generator to get
    \end{itemize}
    \noindent\paragraph{State (shared between all runs)}
    \begin{itemize}
    \item $\mathcal{X}\in\mathcal{P}(\N\times\G)$ (initialized to
      $\emptyset$): generators to avoid
    \end{itemize}
    \noindent\paragraph{Algorithm}
    \begin{enumerate}
    \item $c\leftarrow(p-1)/q$\hfill// typically, $c=2$
    \item $x\leftarrow\shatwo(\verb=ggen|=i)$\hfill// $i$ in base 10,
      output as a big-endian number
    \item $h\leftarrow x^c$
    \item If $h\in\{0,1,g\}$, abort
    \item If $\exists j\neq i, (j,h)\in\mathcal{X}$, abort
    \item $\mathcal{X}\leftarrow\mathcal{X}\cup\{(i,h)\}$
    \item Return $h$\hfill// $h\in\G$
    \end{enumerate}
  \end{framed}
  \caption{Function $\mathsf{GetGenerator}(i)$ (for a multiplicative subgroup of a finite field)}
  \label{get-generator}
\end{table}

\begin{table}
  \begin{framed}
    \noindent\paragraph{Input}
    \begin{itemize}
    \item $\psi=[j_1,\dots,j_N]\in\Psi_N$: permutation
    \item $\textbf{h}=[h_1,\dots,h_N]\in\G^N$: independent generators
    \end{itemize}
    \noindent\paragraph{Algorithm}
    \begin{enumerate}
    \item For $i=1,\dots,N$:
      \begin{itemize}
      \item Pick $r_{j_i}$ at random in $\Z_q$
      \item $c_{j_i}\leftarrow g^{r_{j_i}}\times h_i$
      \end{itemize}
    \item $\textbf{c}\leftarrow[c_1,\dots,c_N]$
    \item $\textbf{r}\leftarrow[r_1,\dots,r_N]$
    \item Return $(\textbf{c},\textbf{r})$\hfill//
      $\textbf{c}\in\G^N$, $\textbf{r}\in\Z_q^N$
    \end{enumerate}
  \end{framed}
  \caption{Function $\mathsf{GenPermutationCommitment}(\psi,\textbf{h})$}
  \label{gen-permutation-commitment}
\end{table}

\begin{table}
  \begin{framed}
    \noindent\paragraph{Input}
    \begin{itemize}
    \item $\textbf{e}=[e_1,\dots,e_N]\in\texttt{ciphertext}^N$: array
      of ciphertexts, or
    \item $\textbf{c}=[c_1,\dots,c_N]\in\G^N$: array of group elements
    \end{itemize}
    \noindent\paragraph{Algorithm}
    \begin{enumerate}
    \item set $S$ to the empty string
    \item For $i=1,\dots,N$:
      \begin{itemize}
      \item append $\alphalabel(e_i)$, a comma, $\betalabel(e_i)$ and a comma to $S$, or\hfill// in base 10
      \item append $c_i$ and a comma to $S$\hfill// in base 10
      \end{itemize}
    \item Return $S$\hfill// $S\in\jstring$
    \end{enumerate}
  \end{framed}
  \caption{Functions $\llbracket\textbf{e}\rrbracket$ and $\llbracket\textbf{c}\rrbracket$}
  \label{serialize-array}
\end{table}

\begin{table}
  \begin{framed}
    \noindent\paragraph{Input}
    \begin{itemize}
    \item $N\in\N$: number of ciphertexts
    \item $S\in\jstring$: challenge string
    \end{itemize}
    \noindent\paragraph{Algorithm}
    \begin{enumerate}
    \item $H\leftarrow\shatwo(S)$\hfill// output interpreted as an
      hexadecimal string
    \item For $i=0,\dots,N-1$:
      \begin{enumerate}
      \item $T\leftarrow\shatwo(i)$\hfill// input taken as decimal,
        output interpreted as hexadecimal
      \item $u_i\leftarrow\shatwo(HT)\mod q$\hfill// output
        interpreted as big-endian
      \end{enumerate}
    \item $\textbf{u}\leftarrow[u_0,\dots,u_{N-1}]$
    \item Return $\textbf{u}$\hfill// $\textbf{u}\in\Z_q^N$
    \end{enumerate}
  \end{framed}
  \caption{Function $\mathsf{GetNIZPKChallenges}(N,S)$}
  \label{get-nizkp-challenges}
\end{table}

\begin{table}
  \begin{framed}
    \noindent\paragraph{Input}
    \begin{itemize}
    \item $c_0\in\G$: initial commitment
    \item $\textbf{u}=[u_1,\dots,u_N]\in\Z_q^N$: public challenges
    \end{itemize}
    \noindent\paragraph{Algorithm}
    \begin{enumerate}
    \item For $i=1,\dots,N$:
      \begin{enumerate}
      \item Pick $r_i$ at random in $\Z_q$
      \item $c_i\leftarrow g^{r_i}\times c_{i-1}^{u_i}$
      \end{enumerate}
    \item $\textbf{c}\leftarrow[c_1,\dots,c_N]$
    \item $\textbf{r}\leftarrow[r_1,\dots,r_N]$
    \item Return $(\textbf{c},\textbf{r})$\hfill//
      $\textbf{c}\in\G^N$, $\textbf{r}\in\Z_q^N$
    \end{enumerate}
  \end{framed}
  \caption{Function $\mathsf{GenCommitmentChain}(c_0,\textbf{u})$}
  \label{gen-commitment-chain}
\end{table}

\begin{table}
  \begin{framed}
    \noindent\paragraph{Input}
    \begin{itemize}
    \item $S\in\jstring$: challenge string
    \end{itemize}
    \noindent\paragraph{Algorithm}
    \begin{enumerate}
    \item $c\leftarrow\shatwo(S)\mod q$\hfill// output interpreted as
      a big-endian number
    \item Return $c$\hfill// $c\in\Z_q$
    \end{enumerate}
  \end{framed}
  \caption{Function $\mathsf{GetNIZPKChallenge}(S)$}
  \label{get-nizkp-challenge}
\end{table}

\FloatBarrier

\bibliographystyle{abbrv}
\bibliography{references}


\end{document}
